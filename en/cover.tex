
Clojure? Это где скобочки и нет типов? На производстве? Всё верно. Книга
расскажет о том, как строят боевые системы на этом языке: от простого к
сложному, скобка за скобкой.

До сих пор информацию о Clojure можно было найти только в книгах на английском и
в отдельных статьях в интернете. Ситуация меняется: появилась книга о Clojure на
русском, и это не перевод. Автор --- программист, который зарабатывает этим
языком.

Это не очередное введение в Clojure. Вас ждут семь глав с акцентом на практику и
неочевидные вещи, которых нет в учебных материалах. Вся книга от начала до
конца --- личный опыт автора. Код заимствован из настоящих проектов.
