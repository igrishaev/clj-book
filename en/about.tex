
\section*{About This Book}

You are reading a book about the Clojure programming language. It is the modern dialect of Lisp on the JVM platform. 
Clojure differs from obsolete dialects in that it relies on a functional approach and ~data immutability. 
The language was designed to offer simple solutions to complex problems.

This book is originally written in Russian.
%Вы не найдёте
%тяжёлых предложений, в которых слышна английская речь. Вам не придётся читать
%<<маркер>> вместо <<токен>> и~другую нелепицу. Термины написаны в том виде,
%чтобы быть понятными программисту.

Our book has no introduction which would say what things you need to download and install.
We also won't talk about nuts and bolts like numbers and strings. 
There are tons of articles and blog posts on the topic of introduction to Clojure. 
It would be unfair to offer you information, half of which has been said more than once. 
I think, from cover to cover, this book contains things that no one has written about. 

Our book is based on practice and this is one of its advantages. The code examples are taken from real projects. 
The author personally used all the techniques discussed below. Describing the problem, we will take into account what tasks await you in production. 
We'll show you where theory diverges from practice and what you need to prefer in such a situation.

Let me speak briefly about what awaits you. First, let's talk about web development: remember what HTTP protocol is and see how to work with it in Clojure. 
After that we will look at the Clojure.spec library for validating data. 
Chapter 3 tells about exceptions. Chapter 4 turns to mutable data.
Then we'll go to a configuration. Chapter 6 introduces systems. Finally, the last chapter will teach you how to write tests.

Even if you don't like Lisp and have picked up this book by accident, do not rush to put it away. 
Clojure is a different world with new rules, and our book is your chance to get there. 
Maybe Clojure will change your mind about programming. You will find questions where, it would seem, everything has been decided.

We wish you have the patience to read the book to the end.

\newpage

\section*{Acknowledgement}

My acknowledgement to the Flyerbee start-up where I start working with Clojure. 
It was there that I reinforced my theoretical knowledge of this language by practice.

Presently, I am happy to work at Exoscale, surrounded by talented engineers. 
I learned a lot of things, not just technical, in this team.

I would like to thank Peter Maslov and Eugene Klimov for finding lots of typos. 
My thanks to Alexey Shipilov and Dosbol Zhantolin who made important comments on the first and last chapters, respectively. 
All readers of my blog who showed me my mistakes, well done!

%\ifx\publisher\ridero
%Благодарю коллектив издательства Rider\'{o} за то, что взяли рукопись в
%работу. Их усилиями вы читаете эту книгу сейчас.
%\fi

%\ifx\publisher\helicon
%Благодарю коллектив издательства Геликон-Плюс за то, что взяли рукопись в
%работу. Их усилиями вы читаете эту книгу сейчас.
%\fi

\section*{Feedback}

The author will be grateful for pointing out typos and inaccuracies. Please, send them to \emaillink. 
%Возможно, в промежутках между тиражами
%получится обновить макет, и следующий читатель не~увидит ошибки, о~которой вы
%сообщили. 
The author will take into consideration all your words.
