\chapter*{Об этой книге}

У вас в руках книга о языке программирования Clojure. Это современный диалект
Лиспа, который работает на платформе JVM. От устаревших диалектов Clojure
отличается тем, что делает ставку на функциональный подход и неизменяемость
данных. Язык устроен так, чтобы решать сложные задачи простым способом.

Эта книга~--- не перевод. Она изначально написана на русском языке. Вы не
найдете тяжелых предложений, в которых отчетливо слышна английская речь. Вам не
придется читать <<маркер>> вместо <<токен>> и другую нелепицу. Термины написаны
в том виде, чтобы быть понятными программисту.

В этой книге нет вводной части, где написано, что скачать и установить. Также мы
не рассматриваем азы вроде чисел и строк. На тему введения уже написаны другие
книги и посты в блогах. Автор считает, было бы нечестно предлагать материал, где
половина повторяет сказанное ранее. Эта книга от начала и до конца~--- то, о чем
еще никто не писал.

Другое ее достоинство~--- упор на практику. Примеры кода взяты из реальных
проектов. Все техники и приемы автор опробовал лично. В описании проблем мы
отталкиваемся от того, что вас ждет на производстве. Мы покажем, где теория
расходится с практикой и что предпочесть в таком случае.

Коротко о том, что вас ждет. Начнем с веб-разработки~--- вспомним протокол HTTP
и как с ним работать в Clojure. Затем рассмотрим Clojure.spec~--- библиотеку для
проверки данных. Третья глава расскажет про исключения. Четвертая~--- про
изменяемые данные. Далее переходим к конфигурации. В шестой главе знакомимся с
системами. В последней учимся писать тесты.

Даже если книга попала к вам случайно, и вы не любите Лисп, не спешите ее
откладывать. Clojure это другой мир и новые правила, а книга~--- шанс туда
попасть. Может быть, Clojure изменит ваше мнение о программировании. Обнаружит
вопросы там, где, казалось бы, все решено.

Автор будет признателен за указанные опечатки и неточности. Присылайте их по
адресу \spverb|ivan@grishaev.me|. Возможно, в промежутках между тиражами
получится обновить макет, и следующий читатель не увидит ошибки, о которой вы
сообщили. И конечно, автор учтет все замечания при переводе на английский язык.

Желаем читателю терпения, чтобы прочесть книгу до конца.

\section*{Благодарности}

Спасибо стартапу Flyerbee, моей первой работе на Clojure. Свои скромные знания я
обкатывал именно там.

Я счастлив работать в компании Exoscale в окружении талантливых
инженеров. Многие вещи, не только технические, я впервые узнал в этом
коллективе.

Спасибо Петру Маслову за крупную партию найденных опечаток. Досбол Жантолин внес
важное замечание к последней главе, а читатель Antony~--- к первой. Молодцы все,
кто указал на ошибки в комментариях в блоге.

Благодарю коллектив издательства Геликон-Плюс за то, что взяли рукопись в
работу. Их усилиями вы читаете эту книгу сейчас.
